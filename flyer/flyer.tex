% !TeX root = flyer.tex
% vim: set et sw=2 ts=2 spell:
\documentclass{article}

\usepackage[fixed]{fontawesome5}
\usepackage{roboto}
\usepackage{roboto-mono}
% \usepackage{beramono}

\usepackage[a5paper, hmargin = 2.5cm, vmargin=2cm]{geometry}

\usepackage{hyperref}
\hypersetup{hidelinks = true, colorlinks = false,}

\usepackage{tikz}
\usetikzlibrary{positioning}

\usepackage{amsmath}
\usepackage{amssymb}

\renewcommand{\vec}[1]{\mathbf{#1}}
\DeclareMathOperator{\dotp}{\mathbf{\cdot}}

\begin{document}
  \thispagestyle{empty}
  \vfill

  % Background

  \begin{tikzpicture}[
      remember picture,
      overlay,
      text = lightgray,
      font = \bfseries,
      every node/.style = {
        scale = 3,
      },
    ]

    \useasboundingbox (current page.north west) -- (current page.south east);
    \coordinate (O) at (current page.center);

    \node[rotate = 20, xshift = 5mm, yshift = 32mm] at (O) {\(
      \displaystyle
      \vec{\nabla}^2 \vec{H} 
        - \mu\sigma \frac{\partial\vec{H}}{\partial t}
        - \mu\varepsilon \frac{\partial^2 \vec{H}}{\partial t^2}
      = \vec{0}
    \)};

    \node[rotate = -10, xshift = 10mm, yshift = 19mm] at (O) {\(
      \displaystyle
      H = - \sum_{x \in \mathbb{X}} p(x) \log p(x)
    \)};

    \node[rotate = 10, xshift = 25mm, yshift = 5mm] at (O) {\(
      \displaystyle
      \oint_{\partial S} \vec{E} \dotp d\vec{l}
        = -\frac{d}{dt} \int_S \vec{B} \dotp d\vec{s}
    \)};

    \node[rotate = -5, xshift = -20mm, yshift = 3mm] at (O) {\(
      \displaystyle
      X(\Omega) = \int_\mathbb{R} x(t) e^{j\Omega t} \,dt
    \)};

    \node[scale = .7, rotate = 3, xshift = 15mm, yshift = -12mm] at (O) {\(
      \displaystyle
      \nabla^2 f = \nabla \cdot \nabla f = \sum_{i = 1}^n \frac{\partial^2 f}{\partial x_i^2}
    \)};

    \node[scale = .7, rotate = 5, xshift = 25mm, yshift = -30mm] at (O) {\(
      \displaystyle
      \sum_{i < j} \begin{vmatrix}
        u_i & v_i \\
        u_j & v_j
      \end{vmatrix} \vec{e}_i \vec{e}_j
      = \vec{u} \wedge \vec{v}
    \)};

    \node[scale = .6, rotate = -8, xshift = -30mm, yshift = -18mm] at (O) {\(
      \displaystyle
      \sigma_2 = \begin{bmatrix}
        0 & -i \\
        i & 0 \\
      \end{bmatrix}
    \)};

    \node[rotate = 20, xshift = -30mm, yshift = -15mm] at (O) {\(
      \displaystyle
      E | \Psi \rangle =
      \hat{\mathrm{H}} | \Psi \rangle
    \)};

    \node[scale = .6, rotate = -5, xshift = -15mm, yshift = -55mm] at (O) {\(
      \displaystyle
      f'(x) = \lim_{h \to 0} \frac{f(x + h) - f(x)}{h}
    \)};

    \node[scale = .8, rotate = 5, xshift = 20mm, yshift = -42mm] at (O) {\(
      \displaystyle
      \|f\|_\infty = \sup_{x \in [a,b]} | f |
    \)};
  \end{tikzpicture}

  % Content

  \begin{center}
    {\Huge\bfseries \LaTeXe{} \textsf{Workshop}} \\[4mm]
    {\sffamily\large Fachschaft Elektrotechnik}
  \end{center}

  \vspace{10mm}

  \sffamily\large

  \begin{itemize}
    \item[\faIcon{calendar-check}] 1. October --- 17 to \~{}19
    \item[\faIcon{map-marker-alt}] Campus Rapperswil Room 5.002
    \item[\faIcon{github}] \href{https://github.com/OpenHSR/LaTeX-Workshop}{\texttt{OpenHSR/LaTeX-Workshop}}
    \item[\faIcon{laptop}] Bring your laptop!
  \end{itemize}

  \vspace{15mm}

  {\rmfamily\noindent
    Does this font look familiar to you?
    It's \LaTeX{}, the standard typesetting tool in Academia!
    If you want your Bachelor thesis to look any good, you have to learn how to use it!
  }

  \vspace{15mm}

  \begin{itemize}
    \item[\faIcon{users}] For beginners to intermediate
    \item[\faIcon{comments}] Presentation + Help to get you started
    \item[\faIcon{flask}] Workshop content
      \begin{itemize}
        \item Absolute essentials
        \item Typeset mathematics
        \item Bibliography management
      \end{itemize}
    \item[\faIcon{fast-forward}] Extras (if there is time): Source code listings. Drawings with \textrm{Ti\textit{k}Z}. Plot data with PGFPlots.
  \end{itemize}

  \vfill
\end{document}

