\documentclass[a4paper]{article}

% preamble

%% packages
\usepackage[english]{babel} % change language (dates, captions, ...)

\usepackage{amsmath} % most mathematical symbols
\usepackage{amssymb} % even more symbols (see refs/symbols.pdf)
\usepackage{esint}   % contour integrals

\usepackage[%
    top = 3cm, bottom = 2.5cm, left = 3cm, right = 3cm,
]{geometry} % change margins
\usepackage{fancyhdr} % change headers

%% clear header and footers
\fancyhf{}
\fancyhead[R]{\textsc{Quick reference}}
\fancyhead[L]{\textsf{Open\textcolor{cyan!50!blue}{\textbf{\textbackslash{HSR}}}}}
\fancyfoot[C]{\thepage}
%% make line a bit thicker
\renewcommand{\headrulewidth}{0.5pt}
%% apply style
\pagestyle{fancy}

\usepackage[
    colorlinks=true,
    linkcolor=.,
    citecolor=.,
    urlcolor=blue
]{hyperref} % highlight urls in blue

\usepackage{xcolor}      % use colors
\usepackage{float}       % exact placing of figures

\usepackage{booktabs} % nicer looking table separators
\usepackage{multirow} % cells that span multiple rows
\usepackage{multicol} % split text into columns (like a newspaper)
\usepackage{array}    % more powerful table configuration

\usepackage{graphicx}    % include graphics
\usepackage{tikz}        % draw graphics yourself
\usepackage{tikz-timing} % draw timing diagrams 
\usepackage{pgfplots}    % beautiful plots
\pgfplotsset{compat=newest}

\usepackage{framed}   % simple frame
\usepackage{mdframed} % fancy frames

%% a somewhat basic frame
\mdfdefinestyle{mystyle}{%
    linecolor=black,%
    linewidth=1.5pt,%
    frametitlerule=true,
    frametitlebackgroundcolor=gray!20,
    innertopmargin=\topskip,
}

\usepackage{listings} % show code

%% create a lstlisting style
\lstdefinestyle{latex}{
    belowcaptionskip=\baselineskip,
    breaklines=true,
    frame=none,
    inputencoding=utf8,
    % margin
    xleftmargin=\parindent,
    % numbers
    numbers=left,
    numbersep=5pt,
    numberstyle=\ttfamily\footnotesize\color{gray},
    % background
    backgroundcolor=\color{gray!10},
    % default language:
    language=[LaTeX]TeX,
    showstringspaces=false,
    % font
    basicstyle=\ttfamily\small,
    identifierstyle=\color{black},
    keywordstyle=\color{green!40!black},
    commentstyle=\color{gray},
    stringstyle=\color{orange},
}

%% and set the chosen style
\lstset{style=latex}



%% metadata
\title{A few packages to make your document look better}
\author{Naoki Pross}
\date{\today}


% document starts here
\begin{document}
\maketitle

\section{\texttt{babel}}
\begin{lstlisting}
\usepackage[english]{babel} % change language (dates, captions, ...)
\end{lstlisting}


\section{\texttt{amsmath}, \texttt{amssymb}, \texttt{esint}}
These packages add more math symbols.
\[
    \oint_{\partial V} \mathbf{E}_0 \cdot \mathrm{d}\mathbf{S} 
	= \int_V \nabla \cdot \mathbf{E}_0 \, \mathrm{d}V
\]
\begin{lstlisting}
\oint_{\partial V} \mathbf{E}_0 \cdot \mathrm{d}\mathbf{S} 
    = \int_V \nabla \cdot \mathbf{E}_0 \, \mathrm{d}V
\end{lstlisting}


\section{\texttt{geometry}, \texttt{fancyhdr}}
You are looking at a document with the following settings:
\begin{lstlisting}
\usepackage[%
    top = 3cm, bottom = 2.5cm, left = 3cm, right = 3cm,
]{geometry} % change margins
\end{lstlisting}

\begin{lstlisting}
\usepackage{fancyhdr} % change headers
%% clear header and footers
\fancyhf{}
\fancyhead[R]{\textsc{Quick reference}}
\fancyhead[L]{\textsf{Open\textcolor{cyan!50!blue}{\textbf{\textbackslash{HSR}}}}}
\fancyfoot[C]{\thepage}
%% make line a bit thicker
\renewcommand{\headrulewidth}{0.5pt}
%% apply style
\pagestyle{fancy}
\end{lstlisting}


\section{\texttt{xcolor}}
You can now use \textcolor{red}{colors}, and 
\textcolor{red!40!blue}{combination of colors}!

\begin{lstlisting}
You can now use \textcolor{red}{colors}, and 
\textcolor{red!40!blue}{combination of colors}!
\end{lstlisting}


\section{\texttt{float}, \texttt{booktabs}, \texttt{multirow}, \texttt{multicol}, \texttt{array}}
% float allows for exact placement with `H', works with figures too
\begin{table}[H]
    \centering{}
    \caption{A sane looking table}
    % array allows to use >{} and <{} to add a command 
    % before and after the start of a cell
    \begin{tabular}{>{\bfseries}r l l >{\(}l<{\)}}
        % beautiful separators with booktabs
        \toprule
        % multicolumn (and multirow) allows to do what is says, 
        % `c' stands for "center", can also be `r' or `l'
        Bold & \multicolumn{2}{c}{Multiple columns}   & \gamma\\
        \midrule
           A & Hello      & World                     & \psi \\
           B & Welcome to & \LaTeX                    & \varphi \\
        \bottomrule
    \end{tabular}
\end{table}

\begin{lstlisting}
% float allows for exact placement with `H', works with figures too
\begin{table}[H]
    \centering{}
    \caption{A sane looking table}
    % array allows to use >{} and <{} to add a command 
    % before and after the start of a cell
    \begin{tabular}{>{\bfseries}r l l >{\(}l<{\)}}
        % beautiful separators with booktabs
        \toprule
        % multicolumn (and multirow) allows to do what is says, 
        % `c' stands for "center", can also be `r' or `l'
        Bold & \multicolumn{2}{c}{Multiple columns}   & \gamma\\
        \midrule
           A & Hello      & World                     & \psi \\
           B & Welcome to & \LaTeX                    & \varphi \\
        \bottomrule
    \end{tabular}
\end{table}
\end{lstlisting}

\section{\texttt{graphicx}}

\section{\texttt{tikz}, \texttt{tikz-timing}, \texttt{pgfplots}}

\end{document}
